\documentclass{amsart}
%%%%%%%%%%%%%%%%%%%%%%%%%%%%%%%%%%%%%%%%%
%version 3 of paper

\usepackage{amscd}
\usepackage{amsmath, calculation}
\usepackage{amssymb}
\usepackage{amsthm}
\usepackage{subfig}
\usepackage{graphicx}
%\usepackage{geometry}
%\usepackage[all,cmtip]{xy}
\usepackage{tikz}
\usetikzlibrary{shapes.geometric, arrows}

%\setlength{\textheight}{9.8in} \setlength{\topmargin}{0.0in}
%\setlength{\headheight}{0.0in} \setlength{\headsep}{0.0in}
%\setlength{\leftmargin}{0.5in}\setlength{\oddsidemargin}{0.0in}
%\setlength{\parindent}{0pc}\linespread{1.6}
%\setlength{\textwidth}{6in}


%%%%%%%%%%%%%%%%%%%%%%%%%%%%%%%%%%%%%%%%%%%%%
\theoremstyle{plain}
\newtheorem{thm}{Theorem}[section]
\newtheorem{lem}[thm]{Lemma}
\newtheorem{prop}[thm]{Proposition}
\newtheorem{cor}[thm]{Corollary}
\newtheorem{rem}[thm]{Remark}
\newtheorem{alem}[thm]{Almost A Lemma}
\newtheorem{lts}[thm]{Left to Show}
\newtheorem*{dfn}{Definition}
\newtheorem*{conj}{Conjecture}



\newenvironment{poc}[1][Proof of Claim]{\begin{trivlist}
\item[\hskip \labelsep {\bfseries #1}]}{\end{trivlist}}



%%%%%%%%%%%%%%%%%%%%%%%%%%%%%%%%%%%%%%%%%%%%%


\newcommand{\GL}{\Gamma_L}
\newcommand{\GLone}{\Gamma_{L'}}
\newcommand{\Gnot}{\Gamma_0}
\newcommand{\GK}{\Gamma_K}
%\newcommand{\GDS}{\Gamma_{D_s}}
%\newcommand{\GDF}{\Gamma_{D_f}}
\newcommand{\GDS}{\Gamma_{s}}
\newcommand{\GDF}{\Gamma_{f}}


\newcommand{\NGL}{N(\Gamma_L)}
\newcommand{\SNL}{\Sth - L}
\newcommand{\CGL}{Comm^+(\Gamma_L)}
\newcommand{\Hth}{\mathbb{H}^3}
\newcommand{\PSLTC}{PSL(2, \mathbb{C})}
\newcommand{\PSLTF}{PSL(2, \mathbb{F})}
\newcommand{\PSLTZI}{PSL(2, \mathbb{Z}[i])}
\newcommand{\SLTC}{SL(2, \mathbb{C})}
\newcommand{\TT}{\mathcal{T}}
\newcommand{\Z}{\mathbb{Z}}
\newcommand{\ZnZ}{\mathbb{Z}/n\mathbb{Z}}
\newcommand{\ZthZ}{\mathbb{Z}/2\mathbb{Z}}
\newcommand{\ZFZ}{\mathbb{Z}/4\mathbb{Z}}
\newcommand{\Q}{\mathbb{Q}}
\newcommand{\Qy}{\mathbb{Q}(y)}


\newcommand{\TtwoxI}{\mathbb{T}^2 \times I}
\newcommand{\PxS}{\mathbb{P} \times \mathbb{S}^1}
\newcommand{\RPth}{RP^3}
\newcommand{\Sth}{S^3}
\newcommand{\SxD}{\mathbb{S}^1 \times D}
\newcommand{\SxStwo}{\mathbb{S}^1 \times S^2}


\begin{document}
\title{Maximal injectivity radius of an ideal tetrahedron}
\author{G\"{o}rner, Haraway, Hoffman, Trnkova}
\date{\today}
\maketitle


We work primarily in the Beltrami-Klein model. However,
since SnapPy parametrizes the isometry class of an
ideal tetrahedron by elements of the upper half-plane,
we need to calculate a map from $\mathbf{C}$ to
the sphere which takes circles to circles. The following
is the usual stereographic projection $\phi$ to the unit sphere
from $(0,0,1)^t$:
\[
\phi = a+b\cdot i \mapsto
\begin{pmatrix}
2\cdot a / s\\
2\cdot b / s\\
(s-2)/s
\end{pmatrix},
\]
where $s = a^2 + b^2 + 1$.
We note that $\phi.(0) = (0,0,-1)^t$,
$\phi.(1) = (1,0,0)^t$, $\phi.(-1) = (-1,0,0)^t$,
$\phi.(i) = (0,1,0)^t$, and $\phi.(-i) = (0,-1,0)^t$.
Thus the involutions $w\mapsto -1/w$, $w\mapsto -w$, 
and $w \mapsto 1/w$ push forward through $\phi$
to the involutions changing the sign of two coordinates.

We can make these involutions be the half-revolutions
around the perpendiculars through opposite edges in our
tetrahedron by applying an isometry, the image
of $z$ under which is $\sqrt{(2-z-2\cdot\sqrt{1-z})/z}$.
(We can check this using a computer algebra system.)
Let $a+b\cdot i = \sqrt{(2-z-2\cdot\sqrt{1-z})/z}$.

Let us write $\phi.(a+b\cdot i) = (C,D,E)^t$ and
call it $w$. The image of $w$ under the involutions
are the other vertices of the tetrahedron, viz.
$(C,-D,-E)^t$, $(-C,D,-E)^t$, and $(-C,-D,E)^t$.
The triangle these three form is the 
side $S$ of the tetrahedron opposite $w$.
The image of our original tetrahedron is the convex
hull of these four points on the unit sphere.

We first would like to show that the point of
maximum injectivity radius is, indeed, the center
of the Beltrami-Klein model, the origin $O = (0,0,0)^t$.
The involutions have $(0,0,0)^t$ as their unique
common fixed point. For the action of the Klein four-group
they generate, we can take as fundamental domain the
cone $C$ from $S$ to $O$. Now, $C$ is precisely those
points in the tetrahedron closest to $S$. So it will
suffice to calculate the point(s) in $C$ with maximum
distance from $S$. Now the locus of points distance less than
$r$ to $S$ is a strictly convex, round domain $D_r$.
Let $R$ be the distance from $O$ to $S$. Consider
that the closure of $D_R$ contains $O$ and the vertices
of $S$. By roundness and strict convexity, $\overline{D_R} \cap C$
is just the vertices of $C$. In particular, the only point
of $C$ on the boundary of $\overline{D_R}$ is $O$, which
therefore is the only point distance exactly $R$ from $S$,
and all other points in $C$ have distance less than $R$.
Thus, $O$ is the unique point in $C$ of maximal injectivity
radius. By symmetry, $O$ is the unique point in the
tetrahedron of maximal injectivity radius.

Having found a (unique) point of maximum injectivity
radius, we now calculate this radius. First,
note that the perpendicular from $O$ to a plane
in the Beltrami-Klein model is, in fact, a
Euclidean perpendicular in the (incomplete) metric the ball
inherits from Euclidean space. Furthermore,
suppose that the Euclidean distance from $O$ to
the plane is $p$. Then the hyperbolic distance
from $O$ to the plane is $(1/2) \cdot \log (\frac{1+p}{1-p})$.
But the derivative of this expression with respect
to $p$ is $1/(1-p^2)$, which is greater than 1 for $0<p<1$.
So $p$ is less than this expression. In fact, $p$
is the first Taylor approximation of this expression
near $p=0$. So just calculating the Euclidean distance
from $O$ to the plane is a good enough approximation for us.
But if one so desires, it is simple to calculate
the true injectivity radius by the above expression.

The calculation goes as follows. We can express
the plane supporting $S$ as $\eta^{-1}.(-C\cdot D \cdot E)$
where $\eta$ is the linear functional $(D\cdot E,\ C\cdot E,\ C\cdot D)$.
The unit normal to this plane, therefore, is
$\hat{\xi} = (1/k) \cdot \eta^t$, where
$k = \|\eta\|$.
The line through $\hat{\xi}$ and $O$ is
the perpendicular from $O$ to the plane.
It intersects the plane at $(-C\cdot D\cdot E / k) \cdot \hat{\xi}$,
whose norm is, plainly, $|C \cdot D \cdot E| / k$. This norm
is the desired Euclidean distance. We now calculate.
\begin{calculation}[=]
p
\step{above discussion}
|C \cdot D \cdot E| / k
\step{definition of $k$}
|C \cdot D \cdot E| / \sqrt{D^2\cdot E^2+C^2\cdot E^2+C^2\cdot D^2}
\step{definition of $C$, $D$, and $E$; algebra}
|2\cdot a \cdot 2 \cdot b \cdot (s-2) / s^3|
/ \sqrt{(4\cdot b^2 \cdot (s-2)^2 + 4 \cdot a^2 \cdot (s-2)^2 + 16 \cdot a^2 \cdot b^2)/s^4}
\step{algebra}
4\cdot s^2 \cdot |a\cdot b \cdot (s-2)|
/ \left ( s^3 \cdot \sqrt{4 \cdot (b^2 \cdot (s-2)^2 + a^2 \cdot (s-2)^2 + 4 \cdot a^2 \cdot b^2)}\right )
\step{algebra}
2\cdot |a\cdot b \cdot (s-2)|
/ \left ( s \cdot \sqrt{b^2 \cdot (s-2)^2 + a^2 \cdot (s-2)^2 + 4 \cdot a^2 \cdot b^2} \right )
\step{algebra}
2\cdot |a\cdot b \cdot (s-2)|
/ \left ( s \cdot \sqrt{(a^2 + b^2) \cdot (s-2)^2 + 4 \cdot a^2 \cdot b^2} \right )
\step{definition of $s$}
2\cdot |a\cdot b \cdot (a^2+b^2 - 1)|
/ \left ( (a^2+b^2+1) \cdot \sqrt{(a^2 + b^2) \cdot (a^2+b^2-1)^2 + 4 \cdot a^2 \cdot b^2} \right )
\end{calculation}

\end{document}
